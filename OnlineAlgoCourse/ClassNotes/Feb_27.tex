\documentclass[10pt,a4paper]{article}


\usepackage{amsmath, amsthm,amssymb}    % need for subequations
\usepackage{graphicx}   % need for figures
\usepackage{verbatim}   % useful for program listings
\usepackage{color}      % use if color is used in text
\usepackage{hyperref}   % use for hypertext links, including those to external documents and URLs
\usepackage{colortbl}

\setlength{\baselineskip}{16.0pt}  
\setlength{\parskip}{3pt plus 2pt}
\setlength{\parindent}{20pt}
\setlength{\oddsidemargin}{0.5cm}
\setlength{\evensidemargin}{0.5cm}
\setlength{\marginparsep}{0.75cm}
\setlength{\marginparwidth}{2.5cm}
\setlength{\marginparpush}{1.0cm}
\setlength{\textwidth}{150mm}

\newtheorem{thm}{Theorem}[section]
\newtheorem{cor}[thm]{Corollary}
\newtheorem{prop}[thm]{Proposition}
\newtheorem{lem}[thm]{Lemma}
\newtheorem{defin}[thm]{Definition} 

\newcommand{\mU}{\mathcal{U}}
\newcommand{\greedy}{\textsc{Greedy}}

\begin{document}

\title{Online Set Cover \\[1 cm] 
\large{CS 599I  Online Algorithms}
\date{February 27}}  
\maketitle

\section{The Set Cover Problem}

We are given a universe of elements $\mU$, where $|\mU| = n$, and $m$ subsets of $\mU: S_1, \dots, S_m$, such that $\bigcup_i S_i = \mU$. A {\em set cover} for this instance is defined as any $I \subseteq [m]$ such that $\bigcup_{i \in I} S_i = \mU$. The goal of the problem is to find the set $I$ of minimum size. 

This is the {\em unweighted} version of set cover, since each set has an equal weight of 1. It is known that the offline instance of Set Cover is NP-hard.

\section{Offline Approximation Algorithms}

We present two algorithms for the offline version of Set Cover that both give a $O(\log n)$ approximation ratio. 

\subsection{The Greedy Algorithm}

The first algorithm, \greedy, works by greedily choosing the set that covers most of the elements that are still not covered by any set. More formally:

\begin{enumerate}
\item $I \leftarrow \{\}$
\item While $\mU \neq \emptyset$:
\begin{enumerate}
\item Pick $j = \arg \max_i |S_i|$ and let $I \leftarrow I \cup \{j\}$.
\item $\mU \leftarrow \mU \setminus S_j$.
\item $\forall i \in [m]: S_i \leftarrow S_i \setminus S_j$.
\end{enumerate}

\end{enumerate}


\begin{thm}
\greedy is an $O(\log n)$ approximation algorithm for Set Cover.
\end{thm}

\begin{proof}
(sketch) Let $T_1, \dots, T_r$ be the sets picked by \greedy. Let us also denote by $t_i$ the number of elements covered for the first time by the set $T_i$, where $i=1, \dots, r$. Finally, assume that the optimum solution (OPT) has size $O$.

At the first iteration, since \greedy chooses the set of maximum size, $t_1$ will be at least as large as the average set size chosen by OPT.  Hence, $t_1 \geq n/O$. Following similar reasoning, for $i=1, \dots, r$, we have that $t_i \geq (n-\sum_{j=1}^{i-1} t_j) /O$. After some calculations, we conclude that indeed $r \leq O(\log n) O$.
\end{proof}

\subsection{The LP Algorithm}

The second algorithm is based on linear programming and randomized rounding. First, we express Set Cover as an integer linear program. For this, we use an indicator variable $x_i$ which denotes whether set $S_i$ is chosen in the cover (value 1) or not (value 0). 

\begin{align*}
\text{minimize} \quad &  \sum_{i \in [m]} x_i \\
\text{subject to} \quad & \forall e \in \mU : \sum_{i: e \in S_i} x_i \geq 1 \\
& \forall i \in [m]: x_i  = \{0,1\}
\end{align*}

We obtain the relaxed version of the program by allowing $x_i$ to take any real value between $[0,1]$; hence, for the LP the last condition becomes $x_i \geq 0$. By solving the relaxed LP, we obtain a fractional solution $x^{*}$, where $\sum_{i \in [m]} x^{*}_i \leq OPT$. We then round the fractional solution to obtain an integer one. The algorithm, \textsc{Randomized Rounding}, can be summarized as follows:

\begin{enumerate}
\item Solve the relaxed LP to obtain the optimum solution $x^*$. Let $I \leftarrow \emptyset$.
\item For $4 \log n$ times repeat:
\begin{enumerate}
\item $\forall i \in [m]$, let $I \leftarrow I \cup \{i\}$ with probability $x_i^*$.
\end{enumerate}
\end{enumerate}

\begin{thm}
\textsc{Randomized Rounding} gives:
\begin{enumerate}
\item a feasible solution with probability $\geq 1-1/n^2$.
\item $\mathbb{E}[I] = O(\log n) \sum_{i}x_i^* \leq O(\log n) OPT$.
\end{enumerate}
\end{thm}

\begin{proof}
We first show (2). Fix some iteration $k = 1, \dots, 4 \log n$. Let $X_i$ be the indicator random variable that denotes whether we pick set $S_i$ during iteration $k$. We have that $Pr[X_i = 1] = x_i^*$ and  $Pr[X_i = 1] = 1- x_i^*$. If $X = \sum_{i \in [m]} X_i$, by linearity of expectation:
\begin{align*}
\mathbb{E}[X] = \sum_{i \in [m]} \mathbb{E}[X_i] = \sum_{i \in [m]} x_i^*
\end{align*} 
Summing over all rounds, we obtain that the expected size of $I$ will be at most $(4 \log n) \sum_{i \in [m]} x_i^*$.

In order to show (1), we fix some element $f \in \mU$ and some iteration $k$. We then compute the probability that the element is not covered in this iteration:
\begin{align*}
Pr[f \text{ not covered at } k] = \prod_{i: f \in S_i} (1 - x_i^*) \leq \prod_{i: f \in S_i} e^{-x_i^*} = e^{-\sum_{i: f \in S_i} x_i^*} \leq e^{-1}
\end{align*}
where the first inequality comes from the fact that for any $x \geq 0, 1-x \leq e^{-x}$ and the second inequality from the fact that $x^*$ is a feasible solution for the LP. Now, since every iteration has independent choices, the probability that $f$ is not covered at all is:
\begin{align*}
Pr[f \text{ not covered}] \leq (1/e)^{4 \log n} \leq 1/n^3
\end{align*}
We have proven this bound for a fixed element $f$. We next apply the union bound to conclude that the probability that there exists some element $f$ not covered at all is at most $n (1/n^3) = 1/n^2$.
\end{proof}

\section{Online Algorithms}

In the online setting of Set Cover, we are initially given the $m$ sets, but we do not know which elements they contain. At any time $t$, we get a new element $e_t$ and learn which sets contain $e_t$. We then have to irrevocably pick a set that will cover $e_t$ if it is not already covered. The goal is again to minimize the number of sets we pick.

We first give a lower bound on the competitive ratio of Online Set Cover. Consider an instance where $\mU = \{1, \dots, n \}$ and $S_1, \dots, S_m$ are all the sets of size $\sqrt{n}$. The adversary will repeat the following strategy for $\sqrt{n}$ iterations: pick any element that is not covered by the current solution. This implies that any deterministic algorithm will construct a solution with cost $\sqrt{n}$. On the other hand, some set $S_j$ will cover all the elements chosen by the adversary and thus $OPT = 1$. This gives a lower bound of $\sqrt{n}$. However, notice that the number of sets in this case is $m = \binom{n}{\sqrt{n}}$.

We next present an online algorithm with competitive ratio that depends on both $m,n$. As a first step, we will give an algorithm for the fractional version of Set Cover. 

\begin{enumerate}
\item For every $i \in [m]$, initialize $x_i \leftarrow 1/m^2$.
\item At time $t$, when $e_t$ arrives
\begin{enumerate}
{\color{blue}  \item Set $y_e \leftarrow 0$.}
\item While $\sum_{i: e_t \in S_i} x_i < 1$: 
\begin{itemize}
\item $\forall i \text{ s.t.} e_t \in S_i: x_i \leftarrow 2 x_i$
{\color{blue}  \item $y_{e_t} \leftarrow y_{e_t} +1$.}
\end{itemize}
\end{enumerate}
\end{enumerate}

The lines in blue explain how we construct in parallel with the primal solution for the LP a solution for the dual LP, which is the following:

\begin{align*}
\text{maximize} \quad &  \sum_{e \in \mU} y_e \\
\text{subject to} \quad & \forall i \in [m] : \sum_{e \in S_i} y_e \leq 1 \\
& \forall e \in \mU: y_e \geq 0
\end{align*}

The dual variables will only be used for the analysis of the algorithm. 

\begin{thm}
The Online Fractional algorithm has a $O(\log m)$ competitive ratio.
\end{thm}

\begin{proof}
The analysis is based on a primal-dual argument. In particular, we will show how to construct a feasible dual solution $y^*$ such that the total cost of the algorithm is at most $O(\log m) \sum_{e} y_e^*$. Since the cost of a feasible dual is at most the cost of the optimal primal solution, this proves the theorem.
To show this, we will prove two claims. Let $\mU_t$ be the set of elements that appear through time $t$.

\noindent \textbf{Claim 1.} At any time $t$, $\sum_{i=1}^m x_i \leq \sum_{e \in \mU_t} y_e + 1/m$.

We will show this using induction. Before the first iteration, we have that $\sum_{i=1}^m x_i = \sum_{i=1}^m 1/m^2 = 1/m$, whereas $\sum_{e \in \mU_t} y_e= 0$; so this holds with equality. Now, consider some time $t$. If the constraint is satisfied, then nothing changes. Otherwise, we have that $\sum_{i: e_t \in S_i} x_i^{old} < 1$, where $x_i^{old}$ is the value of the variable before time $t$. In this case, the new value will be $x_i  = 2 x_i^{old}$. Hence, the change on the LHS of the inequality will be $\sum_{i=1}^m (x_i - x_i^{old}) = \sum_{i=1}^m x_i^{old} < 1$, whereas the change on the RHS is exactly 1.

\noindent \textbf{Claim 2.} The solution $y_e^* = y_e / (2 \log m)$ is a feasible dual solution.

To show this claim, we have to prove that for any set $S_i$, $\sum_{e \in S_i} y_e \leq 2 \log m$. Indeed, notice that one of the $y_e$ increases only in the case where the variable $x_i$ doubles. Moreover, once $x_i$ reaches 1, it does not need to double again. Since the starting value of $x_i$ is $1/m^2$, $x_i$ can be doubled at most $O(\log m)$ times.
\end{proof}

Thus, we can compute online a fractional solution with competitive ratio $O(\log m)$. To obtain an online integer solution for Set Cover, we can use the randomized rounding technique from the previous section, which will give an additional $O(\log n)$ factor in the competitive ratio. 

\begin{thm}
The competitive ratio for Online Set Cover is $O(\log n \log m)$.
\end{thm}

\end{document}


